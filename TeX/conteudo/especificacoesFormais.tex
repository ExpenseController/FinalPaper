\chapter{ESPECIFICAÇÕES FORMAIS}
	Neste capítulo são descritas as especificações formais da estrutura proposta. São descritos os requisitos funcionais e não funcionais buscados no desenvolvimento da aplicação, além das especificações do projeto por meio de diagrama de \textit{deployment}, diagramas de classe, diagrama de sequência e diagrama de atividades.
	
\section{Descrição da Aplicação}
	A aplicação proposta neste trabalho servirá apenas como exemplo para o objetivo principal que é o balanceamento de carga. A aplicação é consiste em um controle de finanças pessoais, com a possibilidade de registrar proventos e despesas. Para o balanceamento de carga será preparada uma estrutura utilizando \textit{containers}, separando em três camadas: persistência de dados, regra de negócio e interface \textit{web}.
	
\section{Requisitos}
	A seguir são apresentados os requisitos que devem ser atendidos pela aplicação desenvolvida. Os requisitos funcionais que devem ser atendidos pela aplicação desenvolvida são listados no Quadro \ref{Req-Func} e os não funcionais no Quadro \ref{Req-Nao-Func}
	
	\begin{table}[h]
		\caption{Requisitos Funcionais}
		\label{Req-Func}
		\begin{tabular}{|l|}
			\hline
			\textbf{REQUISITOS FUNCIONAIS} \\ \hline
			RF0: O sistema deve manter etiquetas \\ \hline
			RF1: O sistema deve manter transações \\ \hline
			RF2: O sistema deve manter períodos \\ \hline
			RF3: O sistema deve exibir o valor total de transações do período dividido entre proventos e despesas \\
			 \hline
	 		RF4: O sistema deve exibir o valor total das transações de cada etiqueta do período \\ \hline
			RF5: O sistema deve exibir com base nas transações o valor que há na carteira \\ \hline
			RF6: Transações com data futura não devem afetar o valor da carteira \\ \hline
		\end{tabular}
	\end{table}

	\begin{table}[!h]
		\caption{Requisitos Não Funcionais}
		\label{Req-Nao-Func}
		\begin{tabular}{|l|}
			\hline
			\textbf{REQUISITOS NÃO FUNCIONAIS} \\ \hline
			RNF0: O sistema deve estar no ar 24 horas por dia \\ \hline
			RNF1: O sistema deve separar interface, regras de negócio e persistência em \textit{containers} diferentes \\ \hline
			RNF2: O sistema deve aplicar balanceamento de carga \\ \hline
			RNF3: O sistema deve desenvolver as regras de negócio com a linguagem Java na versão 8\\ \hline
			RNF4: O sistema deve utilizar banco de dados PostgreSQL \\ \hline
			RNF5: O sistema deve utilizar \textit{containers} Docker \\ \hline
			RNF6: O sistema deve utilizar CORBA como \textit{middleware} \\ \hline
		\end{tabular}
	\end{table}

\section{Especificações}
	Nesta seção são apresentados os diagrama de \textit{deployment}, classe e de sequência, elaborados sob a Linguagem de Modelagem Unificada (\textit{Unified Modeling Language}, UML), para especificação do sistema proposto.
	
\subsection{Diagrama de \textit{Deployment}}
	No diagrama de \textit{deployment} apresentado na Figura \ref{Diagrama-Deployment} são demonstrados os componentes presentes na estrutura a ser desenvolvida, cada componente terá uma função e realizarão comunicação entre si utilizando o \textit{middleware} CORBA.
	
	A estrutura é separada em três partes: \textit{Application, Controller Server e Database Server}. A primeira é responsável por realizar a interação com o usuário, apresentar a interface visual e executar as requisições solicitadas pelo usuário. O \textit{Controller Server} recebe as solicitações do usuário e faz o manuseio dos dados, validando as informações e aplicando as regras de negócio necessárias. Por fim, temos o \textit{Database Server} que tem como função exclusiva a persistência de dados, ele salva e lê o que for solicitado pelo \textit{Controller}.
	
	\begin{figure}[htb]
		\caption{Diagrama de \textit{Deployment}}
		{\parbox{6cm}{
				\includegraphics[width=14cm]{images/DeploymentDiagram.png}
				\label{Diagrama-Deployment}
				\fonte{Acervo do autor}
		}}
	\end{figure}

\subsection{Diagrama de Classe}
	As classes que compõem o sistema, bem como as relações entre cada classe, são apresentadas
	na Figura \ref{Diagrama-Classe} e servem também para a modelagem do \textit{middleware}.
	
	As classes \textit{Transaction, Tag} e \textit{Period} servem para armazenamento e transição de informações, não apresentam nenhum método. O manuseio dos dados é executado pelas classes \textit{TagController} e \textit{PeriodController}, que por sua vez não necessitam de atributos.
	
	\begin{figure}[!htb]
		\caption{Diagrama de Classe}
		{\parbox{6cm}{
				\includegraphics[width=10cm]{images/ClassDiagram.png}
				\label{Diagrama-Classe}
				\fonte{Acervo do autor}
		}}
	\end{figure}

\subsection{Diagramas de Sequência}
	Os diagramas de sequência das figuras \ref{SeqTag}, \ref{SeqPeriod} e \ref{SeqTransaction} apresentam os fluxos básicos da aplicação. A partir deles é possível conhecer as principais funções do sistema.
	
	O processo de criação de \textit{Tag} está apresentado na Figura \ref{SeqTag} e consiste na criação por parte do usuário de uma nova instância do modelo \textit{Tag}, preenchimento dos seus dados e envio ao controlador para validação, em caso positivo é solicitado ao servidor de persistência para que a nova \textit{Tag} seja salva e uma mensagem de sucesso é retornada ao usuário.
	
	\begin{figure}[!htb]
		\caption{Diagrama de Sequência de adição de \textit{tag}}
		{\parbox{6cm}{
				\includegraphics[width=10cm]{images/SequenceDiagramNewTag.png}
				\label{SeqTag}
				\fonte{Acervo do autor}
		}}
	\end{figure}

	A Figura \ref{SeqPeriod} apresenta a criação de um novo período no sistema, este receberá uma descrição e abrirá uma nova lista de transações. O servidor de controle validará as informações e enviará a persistência para que o novo período seja salvo.

	\begin{figure}[!htb]
		\caption{Diagrama de Sequência de adição de Período}
		{\parbox{6cm}{
				\includegraphics[width=10cm]{images/SequenceDiagramNewPeriod.png}
				\label{SeqPeriod}
				\fonte{Acervo do autor}
		}}
	\end{figure}

	Após ter um período criado é possível adicionar transações de ganhos ou despesas a ele. Uma nova transação deverá ter uma descrição, valor e data, podendo ser atribuído uma \textit{tag}. Ao finalizar o cadastro a aplicação repassa a solicitação para o servidor de controle, que valida as informações e envia para persistência. Sempre que o processo ocorrer corretamente a lista de transações é atualizada, bem como o valor na carteira. Este fluxo está presente da Figura \ref{SeqTransaction}.
	
	\begin{figure}[!htb]
		\caption{Diagrama de Sequência de adição de transação}
		{\parbox{6cm}{
				\includegraphics[width=10cm]{images/SequenceDiagramNewTransaction.png}
				\label{SeqTransaction}
				\fonte{Acervo do autor}
		}}
	\end{figure}