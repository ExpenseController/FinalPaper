\chapter{TRABALHOS CORRELATOS}

\section{Fast Transparent Migration for Virtual Machines}
	Este trabalho apresenta um modelo de implementação de sistema utilizando máquinas virtuais para realizar migrações de forma rápida e transparente. Foi o primeiro a conseguir migrar sistemas sem nenhuma modificação e sem modificações no sistema operacional, funcionando para Windows, Linux e outros. A migração ocorre com total transparência, sem que nem os clientes e nem a aplicação consiga saber que o local de execução foi alterado.
	
	Sistemas com grande demanda podem apresentar lentidão ou falha nos serviços que prestam devido a falta de recursos na máquina em que o processamento está ocorrendo. A migração rápida e transparente é capaz de melhorar o uso do serviço com o balanceamento de carga, movendo serviços para máquinas com mais recursos livres.
	
	O artigo descreve com um exemplo como foi capaz de viabilizar a migração transparente de sistemas com máquinas virtuais, sem a necessidade de efetuar alterações no programa ou no sistema operacional. Foram utilizadas medidas de desempenho de centenas de migrações de máquinas virtuais que utilizam padrões industriais de \textit{benchmark}, ou seja, os padrões mais utilizados na indústria de \textit{software}. Também descreve as necessidades de desempenho e recursos durante a migração.
	
	Para executar a migração é definido a máquina virtual que será migrada e qual seu destino, uma cópia do estado atual da memória da máquina é copiado para o destino enquanto o serviço continua rodando em sua origem. Em seguida o estado da máquina virtual é enviado também para finalizar a configuração do novo local. Uma vez que o destino está configurado, assume a execução e o controle da máquina virtual, para então enviar qualquer resto de estado de memória da máquina origem e removê-la de funcionamento.
	
	O uso de máquinas virtuais foi essencial para o objetivo do artigo, a migração de aplicações era falha pela dificuldade de encapsular o estado de execução e também de manter as configurações do sistema operacional, ambos solucionados com o uso de máquinas virtuais. A migração da memória física é o fator essencial para alcançar a transparência apesar de tornar o processo lento, faz com que a aplicação fique menos tempo fora do ar, menos de um segundo de acordo com os testes.
	
\section{LOAD BALANCING ALGORITHM TO IMPROVE RESPONSE TIME ON CLOUD COMPUTING}
	Balanceamento de carga na nuvem está relacionado com alocação de recursos computacionais para máquinas virtuais de forma que as aplicações ali executadas não falhem por falta de recursos e evitando lentidão para o cliente. A melhor prática para realizar o balanceamento de carga na nuvem irá depender da requisição efetuada que pode ser SaaS (\textit{Software as a Service}), PaaS (\textit{Platform as a Service}) ou IaaS (\textit{Infrastructure as a Service}).
	
	Este artigo propôs um algoritmo de balanceamento de carga para máquinas virtuais que busca diminuir o tempo médio de resposta e o tempo médio de processamento dos sistemas na nuvem. O algoritmo foi comparado aos já conhecidos algoritmos Maxmin, \textit{Throttled} e \textit{Avoid Deadlocks} para avaliar os resultados.
	
	Algoritmos dos autores Rashmi, Rajwinder e Hafiz foram utilizados como base de conhecimento, pois já buscavam otimizar o tempo médio de resposta e processamento no balanceamento de carga em máquinas virtuais. 
	
	A máquina virtual com maior disponibilidade e velocidade para completar a tarefa será selecionada a partir dos parâmetros: lista da carga de trabalho do sistema, lista de tarefas já submetidas a execução, percentual já utilizado da máquina e o tempo de execução esperado virtual. 
	
	Os testes foram realizados com trinta tarefas a serem executadas, uma central de dados e três máquinas virtuais. Comparando ao algoritmo Throttled os tempos médios de execução e resposta foram inferiores no algoritmo desenvolvido, demonstrando a efetividade do mesmo.
	
	O algoritmo Throttled tem como foco a quantidade de carga que uma máquina virtual tem, porém em ambientes na nuvem o poder de processamento de carga é heterogêneo e cada máquina virtual pode ter diferentes custos de tempo para execução. Para um melhor balanceamento o ideal é utilizar o tempo que será gasto para executar as tarefas como principal parâmetro, este foi o diferencial que tornou o algoritmo proposto superior ao Throttled.

\section{Methods and apparatus for performing dynamic load balancing of processing resources}
	O artigo apresenta uma forma de alocar recursos de processamento para requisições de rede utilizando balanceamento de carga automático sem uma pré-configuração manual. Balanceamento de carga refere-se a distribuição de requisições de rede para diferentes servidores, de forma que todos consigam executar sem sobrecarga, garantindo disponibilidade e maior confiabilidade.
	
	Normalmente o balanceamento de carga é estático, feito por um programador que manualmente faz a programação ou configura um \textit{hardware} para isso. O administrador do sistema é responsável por definir um modelo de balanceamento, que pode ser \textit{Round Robin}, \textit{Least Loaded} ou \textit{Least Busy}. Em casos onde a configuração definida não estiver sendo efetiva, o próprio administrador terá que acessar o sistema e alterar as configurações, o problema é que isso não será feito diariamente, alterar as configurações para melhorar a performance costumam ocorrer uma vez por semana ou por mês, demonstrando que a configuração manual não é a ideal para adaptação de mudanças em tempo real.
	
	O objetivo do artigo é a apresentação de um algoritmo de balanceamento de carga automático, para tal existe um computador com um \textit{software} que será responsável por receber requisições e distribuir estas da melhor maneira entre os componentes de processamento. Antes de mais nada, cada componente de processamento enviará uma requisição de registro para o servidor central, todas as máquinas serão registras em conjunto de informações como: métricas operacionais, largura de banda, portas de rede utilizadas, memória utilizada, processamento utilizado, cache do sistema operacional, quantidade de operações realizadas pelo processador em um determinado período de tempo, etc.
	
	Ao receber uma requisição de um cliente, o servidor irá utilizar os dados obtidos dos componentes registrados e repassar a tarefa ao que melhor se apresentar. Em caso de um componente estar com dificuldades em processar uma tarefa, ele pode requisitar ao servidor central que distribua a carga tarefa entre mais componentes, ou então, redefinir o método de balanceamento para atribuir a tarefa a um componente com melhores condições.
	
\section{Strategies for CORBA Middleware-Based Load Balancing}
	Este artigo trata de estratégias e arquiteturas utilizadas para o balanceamento de carga utilizando o \textit{middleware} CORBA, apresentando o serviço de balanceamento que desenvolveram e uma comparação de performance com outros serviços existentes no mercado.
	
	É importante conceituar que o balanceamento de carga em sistemas distribuídos pode ser baseado em rede onde o DNS define as rotas, também pode ser baseado em sistemas operacionais sendo capaz de dividir o processamento em mais de uma máquina, ou então, baseado em \textit{middleware} utilizando um componente somente para distribuir o fluxo de requisições entre os componentes. Balanceamentos baseados em rede ou sistemas operacionais costumam sofrer com falta de flexibilidade e adaptação, ao contrário dos \textit{middlewares}.
	
	O \textit{middleware} CORBA apresenta soluções para problemas comuns de sistemas distribuídos como previsibilidade, segurança e tolerância a falhas. Além do mais, conta com diferentes formas de definir qual réplica do servidor será utilizada por um cliente. A vinculação pode ser por sessão e assim todas as requisições seguintes do cliente serão feitas para a mesma réplica até que a sessão seja encerrada. Outra forma é redefinir a réplica a ser utilizada a cada requisição feita, porém o tempo desprendido será maior. A última forma é definir a réplica por demanda, dessa forma o cliente sempre requisita para a mesma réplica até que fique sobrecarregada, neste momento as requisições passam a ser redirecionadas para outra réplica.
	
	A metodologia aplicada foi serviço de balanceamento de carga TAO, que é feito especificamente para CORBA. Essa metodologia garante que as aplicações distribuídas sejam adaptáveis e efetivas, o serviço é capaz de distribuir as requisições por réplicas distintas sem aumentar significativamente o tempo de resposta.
	
	A fase de testes utilizou o tempo de latência para definir se a metodologia utilizada foi superior as já existentes. Inicialmente foram realizadas requisições sem balanceamento de carga para utilizar como referência, e então foram realizados testes com a execução de 200.000 requisições. O resultado final apontou um melhor desempenho na metodologia TAO, com diversas vantagens como escalabilidade e confiabilidade.